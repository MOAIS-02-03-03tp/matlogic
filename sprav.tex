\documentclass[12pt]{article}
\usepackage[a4paper, total={7in,10in}]{geometry}
\usepackage{polyglossia}
\usepackage{ragged2e}
\usepackage{amsmath}
\usepackage{amssymb}
\usepackage{microtype}
\usepackage{graphicx}
\let\ORIincludegraphics\includegraphics
\renewcommand{\includegraphics}[2][]{\ORIincludegraphics[scale=0.65,#1]{#2}}
\usepackage{changepage}
\usepackage{hyperref}
\usepackage{cancel}
\graphicspath{{./images/}}
\setmainlanguage{russian}
\setotherlanguage{english}
\newfontfamily\russianfont[Script=Cyrillic]{Times New Roman}
\newfontfamily\englishfont{Times New Roman}
\setlength{\parindent}{0em}
\setlength{\parskip}{6pt}

\def\posl#1#2{\{#1_{#2}\}}
\DeclareMathOperator*{\sh-like}{\sinh-like}
\DeclareMathOperator*{\ch-like}{\cosh-like}
\DeclareMathOperator*{\th-like}{\tanh-like}
\DeclareMathOperator*{\cth-like}{\coth-like}
\DeclareMathOperator*{\tg-like}{\tan-like}
\DeclareMathOperator*{\ctg-like}{\cot-like}
\DeclareMathOperator*{\arctg-like}{\arctan-like}
\DeclareMathOperator*{\arcctg-like}{\arctan-like}

\begin{document}
    \pagebreak
    \tableofcontents
    \pagebreak
    \section{Теория множеств}
    \justifying
    \subsection{Основные определения}
    \begin{math}
        a \in A\\
        \varnothing\\
        A \subset B <=> A \subseteq B \; and \; \exists x (x \in B i x \not \in A)\\
        A \subseteq B <=> \forall x (x \in A => x \in B)\\
        A=B <=> A \subseteq B \; and \; B \subseteq A\\
        A \land B = {x|x \in A \; and \; x \in B}\\
        A \lor B = {x|x \in A \; or x \; \in B}\\
        A \setminus B={x|x \in A \; and \; x \not \in B}\\
        A \bigtriangleup B = (A \lor B) \setminus (A \land B)=
        (A \setminus B) \lor (B \setminus A)\\
        \bar{A_B}=B \setminus={x \in B | x \not \in A}\\
        A \; \text{x} \; \dots\; \text{x}\; A_n = {(a_1,\dots,a_n)|\forall i \in n (a_i \in A_i)}\\
        \text{Где $(a_1,\dots,a_n)$ - упорядоченный набор который определяется следующим образом}\\
    \end{math}
    \begin{enumerate}
        \item n=0 => $\varnothing$
        \item n=1 => {$a_1$}
        \item n=2 => ${{a_1},{a_1,a_2}}$
        \item n>2 => ($a_1,\dots,a_{n-1},a_n$)=($(a_1,\dots,a_{n-1}),a_n$)
    \end{enumerate}
    $(1,2,3)^{4)}=((1,2),3)^{3)}$=${{(1,2)},{3}}^{3}$={{{1},{1,2}},{(1,2),3}}=
    {{{{1},{1,2}}},{{{1},{1,2}},3}}

    \subsection{Бинарное отношение}
    \underline{Опр} Множество $\rho \subseteq A_1 x\dots x A_n$ называется n-местеыми отношениями на
    $(A_1,\dots,A_n)(n>0)$. Если $A_1=A_2=\dots=A_n(=A)$то $\rho$ называется n-местными отмношениями
    на A. Если n=2, то $\rho$ называется бинарным\\
    \underline{Пример}:\begin{enumerate}
        \item A={1,2,3}
        \item B={a,b}
        \item AxB={(1,a),(2,a),(3,a),(1,b),(2,b),(3,b)}
        \item BxA={(a,1),(a,2),(a,3),(b,1),(b,2),(b,3)}
    \end{enumerate}
    \underline{Опр} Пусть $\rho \subseteq AxB$\\
    Dom($\rho$)={$x\in A |\exists y \in b:(x,y)\in \rho$} - область определения\\
    Im ($\rho$)={$y \in B|\exists x \in A:(x,y)\in \rho$} - мн в значении\\
    $\rho^{-1}$ = {$(y,x)\in BxA|(x,y)\in \rho$} - обратное к $\rho$

    \subsection*{Свойства бинаных отношений на A}
    Пусть $\rho \subseteq A^2$. Тогда\\
    \begin{enumerate}
        \item $\rho$ рефлексивно <=> $\forall x \in A((x,x)\in \rho)$ definition
        \item $\rho$ симметрично <=> $\forall x,y \in A((x,y)\in \rho => (y,x)\in \rho)$
        \item $\rho$ транзитивно <=> $\forall x,y,z \in A ((x,y)\in \rho \; and \; (y,z)\in \rho => (x,z)\in \rho)$
        \item $\rho$ антирефлексивно <=> $\forall x \in A((x,x) \not \in \rho)$
        \item $\rho$ антисимметрично <=> $\forall x,y \in A (x \not = y \; and \; (x,y)\in \rho => (y,x)\not \in \rho)$
    \end{enumerate}
\end{document}