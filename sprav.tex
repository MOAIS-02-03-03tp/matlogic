\documentclass[12pt]{article}
\usepackage[a4paper, total={7in,10in}]{geometry}
\usepackage{polyglossia}
\usepackage{ragged2e}
\usepackage{amsmath}
\usepackage{amssymb}
\usepackage{microtype}
\usepackage{graphicx}
\let\ORIincludegraphics\includegraphics
\renewcommand{\includegraphics}[2][]{\ORIincludegraphics[scale=0.65,#1]{#2}}
\usepackage{changepage}
\usepackage{hyperref}
\usepackage{cancel}
\graphicspath{{./images/}}
\setmainlanguage{russian}
\setotherlanguage{english}
\newfontfamily\russianfont[Script=Cyrillic]{Times New Roman}
\newfontfamily\englishfont{Times New Roman}
\setlength{\parindent}{0em}
\setlength{\parskip}{6pt}

\def\posl#1#2{\{#1_{#2}\}}
\DeclareMathOperator*{\sh-like}{\sinh-like}
\DeclareMathOperator*{\ch-like}{\cosh-like}
\DeclareMathOperator*{\th-like}{\tanh-like}
\DeclareMathOperator*{\cth-like}{\coth-like}
\DeclareMathOperator*{\tg-like}{\tan-like}
\DeclareMathOperator*{\ctg-like}{\cot-like}
\DeclareMathOperator*{\arctg-like}{\arctan-like}
\DeclareMathOperator*{\arcctg-like}{\arctan-like}

\begin{document}
    \pagebreak
    \tableofcontents
    \pagebreak
    \section{Теория множеств}
    \justifying
    \subsection{Основные определения}
    \begin{math}
        a \in A\\
        \varnothing\\
        A \subset B <=> A \subseteq B \; and \; \exists x (x \in B i x \not \in A)\\
        A \subseteq B <=> \forall x (x \in A => x \in B)\\
        A=B <=> A \subseteq B \; and \; B \subseteq A\\
        A \land B = {x|x \in A \; and \; x \in B}\\
        A \lor B = {x|x \in A \; or x \; \in B}\\
        A \setminus B={x|x \in A \; and \; x \not \in B}\\
        A \bigtriangleup B = (A \lor B) \setminus (A \land B)=
        (A \setminus B) \lor (B \setminus A)\\
        \bar{A_B}=B \setminus={x \in B | x \not \in A}\\
        A \; \text{x} \; \dots\; \text{x}\; A_n = {(a_1,\dots,a_n)|\forall i \in n (a_i \in A_i)}\\
        \text{Где $(a_1,\dots,a_n)$ - упорядоченный набор который определяется следующим образом}\\
    \end{math}
    \begin{enumerate}
        \item n=0 => $\varnothing$
        \item n=1 => {$a_1$}
        \item n=2 => ${{a_1},{a_1,a_2}}$
        \item n>2 => ($a_1,\dots,a_{n-1},a_n$)=($(a_1,\dots,a_{n-1}),a_n$)
    \end{enumerate}
    $(1,2,3)^{4)}=((1,2),3)^{3)}$=${{(1,2)},{3}}^{3}$={{{1},{1,2}},{(1,2),3}}=
    {{{{1},{1,2}}},{{{1},{1,2}},3}}

    \subsection{Бинарное отношение}
    \underline{Опр} Множество $\rho \subseteq A_1 x\dots x A_n$ называется n-местеыми отношениями на
    $(A_1,\dots,A_n)(n>0)$. Если $A_1=A_2=\dots=A_n(=A)$то $\rho$ называется n-местными отмношениями
    на A. Если n=2, то $\rho$ называется бинарным\\
    \underline{Пример}:\begin{enumerate}
        \item A={1,2,3}
        \item B={a,b}
        \item AxB={(1,a),(2,a),(3,a),(1,b),(2,b),(3,b)}
        \item BxA={(a,1),(a,2),(a,3),(b,1),(b,2),(b,3)}
    \end{enumerate}
    \underline{Опр} Пусть $\rho \subseteq AxB$\\
    Dom($\rho$)={$x\in A |\exists y \in b:(x,y)\in \rho$} - область определения\\
    Im ($\rho$)={$y \in B|\exists x \in A:(x,y)\in \rho$} - мн в значении\\
    $\rho^{-1}$ = {$(y,x)\in BxA|(x,y)\in \rho$} - обратное к $\rho$

    \subsection*{Свойства бинаных отношений на A}
    Пусть $\rho \subseteq A^2$. Тогда\\
    \begin{enumerate}
        \item $\rho$ рефлексивно <=> $\forall x \in A((x,x)\in \rho)$ definition
        \item $\rho$ симметрично <=> $\forall x,y \in A((x,y)\in \rho => (y,x)\in \rho)$
        \item $\rho$ транзитивно <=> $\forall x,y,z \in A ((x,y)\in \rho \; and \; (y,z)\in \rho => (x,z)\in \rho)$
        \item $\rho$ антирефлексивно <=> $\forall x \in A((x,x) \not \in \rho)$
        \item $\rho$ антисимметрично <=> $\forall x,y \in A (x \not = y \; and \; (x,y)\in \rho => (y,x)\not \in \rho)$
    \end{enumerate}
    \subsection{Отношение эквивалентности}
    \underline{Определение: }Бинарное отношение на множестве называется экивалентным, если оно 
    рефлексивно, симметрично, транзитивно.\\
    \underline{Пример:}\\
    \begin{enumerate}
        \item $\overset{\equiv}{5}$ на $\mathcal{Z}$\\
        x $\overset{\equiv}{5}$ y <=> x-y $\vdots$ 5\\
        \begin{enumerate}
            \item $x-y \vdots 5 => x \overset{\equiv}{5} x$ - рефлексивно
            \item $x \overset{\equiv}{5} y => x-y \vdots 5 => y-x \vdots 5 => y \overset{\equiv}{5}
            x $ - симметрично
            \item $\begin{matrix}
                x \overset{\equiv}{5} y => x-y \vdots 5\\
                y \overset{\equiv}{5} \mathcal{Z} => y - \mathcal{Z} \vdots 5
                \end{matrix} \Bigg\} x-\mathcal{Z} \vdots 5 => x \overset{\equiv}{5} \mathcal{Z}$ - транзитивно 
        \end{enumerate}
        \item A - ин-во все студентов в E702\\
        <x,y> $\in \rho \overset{def}{<=>}$ x и y сидят в одном ряду
        \item A - мн-во $\Delta $-ов пл-ти\\
        $<\Delta_1,\Delta_2> \in \rho \overset{def}{<=>} \Delta_1 \sim  \Delta_2$\\
        $\rho$ - отношение эквивалентности
    \end{enumerate}
    \underline{Определение: } Пусть $\rho$ - отн эквивалентности на A, a $\in$ A\\
    Мн-во $\frac{a}{\rho}=x \in A \Big|<a,x> \in \rho \Big| $ называется классом эквивалентности
    с представлением a.
    \subsubsection*{Теорема 1.3.1}\label{th:1.3.1} 
    $\sqsupset$ $\rho$ - отношение эквивалентности, на A, a,b $\in A$, тогда <a,b> $\in \rho$
    <=> классом эл-ов a и b совпадают $\frac{a}{\rho}=\frac{b}{\rho}$\par\noindent
    \underline{Доказательство:}
    \begin{adjustwidth}{1.5em}{1.5em}
        $\Rightarrow \sqsupset x \in \frac{a}{\rho}$ по опр. <a,x> $\in \rho$. Т.к. <a,b>
        $\in \rho$ и $\rho$ асимметрично, то <b,a> $\in \rho$ $\overset{\text{транз $\rho$}}{\Rightarrow}$
        <b,x>$\in \rho \Rightarrow x \in \frac{b}{\rho} \Rightarrow \frac{a}{\rho}\leq \frac{b}{\rho}$,
        аналогично $\frac{b}{\rho}\leq\frac{a}{\rho} \Rightarrow \frac{a}{\rho}=\frac{b}{\rho}$\\
        $\Leftarrow$ Т.к. $\rho$ рефлексивно, то <a,a> $\in \rho$, т.е. $a \in \frac{a}{\rho}$. Из
        $\frac{a}{\rho}=\frac{b}{\rho} \Rightarrow 2 \in \frac{b}{\rho}$, поэтому <b,a>$\in \rho
        \Rightarrow <a,b> \in \rho$
    \end{adjustwidth}
    \subsection{Отношение порядка}
    \underline{Определение: } Бинарное отношение $\rho$ на мн-ве A называется отношением
    порядка, если оно антиасимметрично и транзитивно.
    \begin{enumerate}
        \item Если $\rho$ рефлексивно то оно называется не строгим
        \item Если $\rho$ антирефлексивно то строгим порядком
        \item Если $\rho$ связно, то линейным порядком
    \end{enumerate}
    \underline{Определение: } Если на мн-ве A введено отношение порядка $\rho$, то A
    называют упорядоченным с помощью $\rho$ мн-вом.\\
    \underline{Пример:} $\leq$ на $\mathcal{Z}$
    \begin{enumerate}
        \item <x,y> $\in \leq <=>(\exists t \in N \; x+\mathcal{Z}=y)$ или x=y
        \begin{enumerate}
            \item $\left.
              \begin{aligned}
                x \leq y \\
                y \leq x
              \end{aligned}
              \right\} \Rightarrow x = y$ - антисимметричность
            \item $\left.
              \begin{aligned}
                x \leq y \\
                y \leq t
              \end{aligned}
              \right\} \Rightarrow x \leq t$ - транзитивность\\
              Оба пункта образуют порядок
            \item $\psi \leq x$ => не строгий порядок
            \item $\psi \not =$ y => $x \leq y$ или $y \leq x$ => линейный
        \end{enumerate}
        \item Рассмотрим на $\rho(\mathcal{A}),$ где $\mathcal{A}=<a,b>$\\
        <x,y> $\in \leq <=> x \leq y,$ где $x,y \in \mathcal{A}$
        \begin{enumerate}
            \item $\begin{aligned}
                x \leq y\\
                y \leq \psi
            \end{aligned}\Bigg\} \psi \leq y$
            \item $\begin{aligned}
                x \leq y\\
                y \leq \mathcal{Z}
            \end{aligned}\Bigg\} x \leq \mathcal{Z}$
            \item $x \leq \psi \Rightarrow$ не строгий порядок
            \item $\{a\} \not = \{b\},$ на $\{a\} \leq \{b\}$ и $\{b\} \leq \{a\} \Rightarrow $ не линейный
        \end{enumerate}
    \end{enumerate}
    \underline{Определение: }$\sqsupset \rho-$ порядок на A, $a \in A,b \in A$
    \begin{enumerate}
        \item a - минимальный $\overset{def}{<=>} \forall x \in A (<x,a> \in \rho \Rightarrow x =a)$
        \item a - максимальный $\overset{def}{<=>} \forall x \in A (<a,x> \in \rho \Rightarrow x =a)$
        \item a - наименьший $\overset{def}{<=>} \forall x \in A (a \not = x \Rightarrow <a,x> \in \rho)$
        \item a - наибольший $\overset{def}{<=>} \forall x \in A (a \not = x \Rightarrow <x,a> \in \rho)$
    \end{enumerate}

    \subsection{Алгебраические операции}%6 параграф, спиздить с 4-5
    \underline{Определение: } Отображение o:$A^n \rightarrow A$ - называется n-местной 
    операцией на A\\
    \underline{Пример:}
    \begin{enumerate}
        \item +: $\mathcal{Z}^n \rightarrow \mathcal{Z}$
        $<-1,5> \rightarrow 4$\\
        $<0,7> \rightarrow 7$
        \item $\lnot: A \rightarrow A, A$ мн-во во всех
        \item x $V_2^2 \rightarrow V_2$\\
        $<\overset{\rightarrow}{a},\overset{\rightarrow}{b}> \rightarrow \overset{\rightarrow}{c}
        ,$ где $\overset{\rightarrow}{c}=\overset{\rightarrow}{a}*\overset{\rightarrow}{b}$
        \item $\ast $: $N^0\rightarrow N$
        ${\oslash } \ast(\oslash)=0$
        \item :: $N^2 \rightarrow N$-не отражения\\
        <5,2> $\rightarrow$ ?
        \item div: $N^2\rightarrow N$\\
        <5,2> $\rightarrow$2
    \end{enumerate}
    \subsection*{Свойства операций}
    Пусть o-двухместная операция на А
    \begin{enumerate}
        \item o коммунитативна $\overset{def}{\Leftrightarrow} \forall x,y \in A(x o y = y o x)$
        \item o ассоциативна $\overset{def}{\Leftrightarrow} \forall x,y,z \in A (xo(yz)=(xoy)z)$
        \item o идентичная отн о $\overset{def}{\Leftrightarrow} e o e = e$
        \item о индепотентная $\overset{def}{\Leftrightarrow} \forall x \in a (x o x =x)$
        \item о нейтральный отн о $\overset{def}{\Leftrightarrow} \forall x \in A(x o e=eox)$
        \item o обратная слева/справа $\overset{def}{\Leftrightarrow} \forall x \in A \exists y \in A:yox=e/xoy=e$
        \item e - нулевой элемент o $\overset{def}{\Leftrightarrow} \forall x \in A(x o e=eox=e)$
        \item o сократима слева/справа $\overset{def}{\Leftrightarrow} \forall x,y,z \in A(xoy=xoz/yox=zox \Rightarrow y=z)$\\
        Пусть * - двфухместная операция на А
        \item * дистрибутивна слева/справа отн о $\overset{def}{\Leftrightarrow} \forall x,y,z \in A(x*(yoz)=(x*y)o(x*z))/
        (yoz)*x=(y*x)o(z*x)$
    \end{enumerate}
    \underline{Определение: } Пусть о - n-местная операция на A, $B \sqsubseteq A$.\\
    Множество B называется замкнутым относительно o если $\forall x,y \in B(xoy \in B)$
    \subsection{Мощность множества}
    \underline{Определение: } Множества А и В называются равномощными, если существует
    биективное отображени $\varphi: A\rightarrow B$\\
    \underline{Обозначение: |A|=|B|}
    \underline{Пример:}
    \begin{enumerate}
        \item $|\mathcal{N}| = |\mathcal{Z}|$\\
        $\mathcal{Z}: ..,-3,-2,-1,0,1,2,3,...$
        \item $|\mathcal{N}|=|\mathcal{Q}|$
        \item $|\mathcal{Z}|=|\mathcal{Q}|$
    \end{enumerate}
    \subsubsection*{Теорема 4(Кантора-Бернштейна)}\label{th:1.7.4}
    |A|=|B| $\Leftrightarrow$ существуют инъективные оторбражения($\varphi: A\rightarrow B$ и $f:B\rightarrow A$)\par\noindent
    \underline{Доказательство:}
    \begin{adjustwidth}{1.5em}{1.5em}
        ($\rightarrow$) |A|=|B| $\overset{\text{по опр}}{\Rightarrow} \exists \varphi A \rightarrow B$ - биекция\\
        $\varphi$ - инъективное отображение\\
        $\varphi^{-1}$ - тоже биекция\\
        $\varphi^{-1}: B\rightarrow A$ инъекция\\
        ($\leftarrow$) Пусть $\varphi: A\rightarrow B $ и $f:B \rightarrow A$ - инъективное отображение\\
        Пусть $a \in A$ Построим цепи=ь с концом a след. образом:
        \begin{itemize}
            \item если у а нет прообразов по f, то будем говорить что цепь с концом а имеет вид а и длину 1
            \item в противном случаем сущ. единств. $b \in B$ такой что f(b)=a
            \item если у b нет  прообразов по $\varphi$, то будем говорить что цепь с концом а имеет вид $a \leftarrow b$
            \item в противном случае существует единственное $a' \in A$ такой что $\varphi(a')=b$
            \item если у a'...
        \end{itemize}
        Этот процесс либо остановится, и мы скажем что цепь с концом а имеет длину $n \in N$, либо нет, и мы скажем
        что цепь с концом а имеет длину $\infty$. Таким образом мы построим цепи для всех элементов множества
        A , и аналогичным образом для B
    \end{adjustwidth}
    Введем обозначения: $A_o \sqsubseteq A$ - множество элементов из А, которые являются концами цепей
    четной длиной\\
    $A_1 \sqsubseteq A$ - с нечетной длиной\\
    $A_\infty \sqsubseteq A$ - с длиной $\infty$\\
    Аналогично $B_0,B_1,B_\infty \sqsubseteq B$\\
    Заметим что $A_0 \lor A_1 \lor A_\infty=A, B_0 \lor B_1 \lor B_\infty =B$\\
    $\forall i,j \in \{0,1,\infty\}(i \not = j \Rightarrow A_i \land A_j= \oslash \; B_i\land B_j=\oslash)$\\
    $f:B_1 \rightarrow A_0$ сюръективно, т.к. $\forall a \in A_0 \exists b \in B_1:f(b)=a$\\
    $\varphi: A_1 \rightarrow B_0$ также сюръективно\\
    Тогда $h: A \rightarrow B$ такое, что для любого $x \in A$

    \section{Хуй пойми какая(списать у народа)}
    \underline{Определение: } Множество функций $F_1,F_2,\dots,F_n,AB$ называется противоречием или невыполнимыми, если $F_1 \bigwedge F_n \equiv F$\\
    \underline{Обозначение}$F_1,\dots,F_n \models $\\
    \underline{Обозначение}$F \equiv \theta \Rightarrow \models$\\
    \subsubsection*{Теорема 8(эквивалентности условия)}\label{th:2.8}
    Следующие условия эквивалентности:\par\noindent
    \begin{enumerate}
        \item $F_1,\dots,F_n \models F$
        \item $F_1,\dots,F_{n+1} \models F_n \rightarrow F$
        \item $\models F_1 \rightarrow(F_2 \rightarrow(\dots \rightarrow (F_n \rightarrow F)))$
        \item $F_1,\dots,F_n \urcorner F \models$
        \item $F_1 \wedge F_n \models F$
        \item $\models F_1 \wedge F_n \rightarrow F$
        \item $F_1 \wedge \dots \wedge F_n \wedge \urcorner F \models$
    \end{enumerate}    
    \underline{Доказательство:}
    \begin{adjustwidth}{1.5em}{1.5em}
        $3 \Longleftrightarrow  6$\\
        Дано $\models F_1 \rightarrow(\dots \rightarrow(F_n \rightarrow F))$\\
        Доказать: $\models F_1 \wedge \dots \wedge F_n \rightarrow F$\\
        \[\blacktriangleleft F_1 \rightarrow(F_2 \rightarrow \dots(F_k \rightarrow F)) \equiv \urcorner F_1
        \vee \urcorner F_2 \vee \dots \vee \urcorner F_n \vee F \equiv (F_1 \wedge \dots F_n) \wedge F \equiv
        F_1 \wedge \dots F_n \rightarrow \equiv \theta \blacktriangleright\]
    \end{adjustwidth}

    \underline{Доказательство:}
    \begin{adjustwidth}{1.5em}{1.5em}
        $1 \Rightarrow 7$\\
        Дано $F_1,\dots,F_n \models F$\\
        Доказать $F_1,\dots, F_n \models$\\
        $\blacktriangleleft$Получим противоречие: существует $\overline{\varepsilon}=\{u,\wedge\}$, такой что \\
        \[F_1(\overline{\varepsilon}) \wedge \dots \wedge F_n(\overline{\varepsilon})\wedge \urcorner F(\overline{\varepsilon})
        = U \Rightarrow F_1(\overline{\varepsilon})=U,\dots,F_n(\overline{\varepsilon})=U \text{ и } \urcorner F(\overline{\varepsilon})
        \Rightarrow F(\overline{\varepsilon})=\wedge\] \\
        Получили противоречие $\blacktriangleright$
    \end{adjustwidth}

    \underline{Доказательство:}
    \begin{adjustwidth}{1.5em}{1.5em}
        $2 \Rightarrow 4$\\
        Дано: $F_1,\dots,F_{n-1}\models F_n \rightarrow F$\\
        Доказать $F_1,\dots,F_n,\urcorner F \models$
        %дописать, маня во флуд моаиса кинула
    \end{adjustwidth}
    \section{7. Метод резолюций в алгебре высказываний}
    \underline{Определение: } Пусть $F \rightleftarrows X \vee \theta_1, \psi \rightleftarrows
    \urcorner X \vee \theta_2$, где X - логическая переменная, $\theta_1,\theta_2$ - 
    эквивалентные дизъюнкции АВ. Резольвентой функции F и $\psi$ называются ф-ла $\theta_1 \vee \theta_2$\\
    \underline{Обозначение} $\text{res}_x(F,\psi)$\\
    \textbf{Лемма(правило резолюций) :}\\
    \[F,\psi \models \text{res}_x(F,\psi)\]\\
    \underline{Доказательство:}
    \begin{adjustwidth}{1.5em}{1.5em}
        \[\text{Пусть }F \rightleftarrows X \vee \theta_1,\psi \rightleftarrows \urcorner X \vee \theta_2\]\\
        \[F \wedge \psi \wedge \urcorner \text{res}_x(F,\psi)\equiv (X \vee \theta_1)\wedge(\urcorner X \vee \theta_2)
        \wedge \urcorner(\theta_1 \vee \theta_2) \equiv (X \vee \theta_1) \wedge (\urcorner X \vee \theta_2) \wedge \urcorner \theta_1 
        \wedge \urcorner \theta_2 \equiv\]\\
        \[X \wedge \urcorner \theta_1 \wedge \urcorner X \wedge \urcorner \theta_2 = \Theta\]
    \end{adjustwidth}
    \underline{Замечание:} Вообще говоря в определении резольвенты,  формула $\theta_1$ или (и) $\theta_2$ могут отсутствовать. В этом случае резольвенты
    (будем называть) $\theta_1,\theta_2$ или специальную формулу, которую будем называть пустым дизъюнктом и обозначать $\varnothing$\\
    \underline{Определение: } Последовательность формул $\psi_1,\dots,\psi_k$ называется резолютивным выводом
    формулы F из множества $\{F_1,\dots,F_n\}$ формул АВ, если для любого $i \leq k$ выполняется условие:\\
    \[\psi_i \in \{F_1,\dots,F_n\} \text{ или } \psi_i \leq \text{res}(\psi_j,\psi_i),\text{ где } j,i < i\]\\
    \[\psi_k \rightleftarrows F.\]\\
    \underline{Пример:}\\
    \begin{math}
        F_1 \rightleftarrows A \vee \urcorner B \vee C\\
        F_2 \rightleftarrows B \vee C\\
        F_3 \rightleftarrows \urcorner A \vee C\\
        F_4 \rightleftarrows \urcorner C\\
        \text{Существует резолютивный вывод } \varnothing  \text{ из } \{F_1,\dots,F_n\} 
    \end{math}

    \begin{math}
        \psi_1 \rightleftarrows A \vee \urcorner B \vee C (\rightleftarrows F_1)\\
        \psi_2 \rightleftarrows \urcorner A \vee C (\rightleftarrows F_3)\\
        \psi_3 \rightleftarrows \text{res}_A (\psi_1,\psi_2) \rightleftarrows \urcorner B \vee C\\
        \psi_4 \rightleftarrows B \vee C (\rightleftarrows F_2)\\
        \psi_5 \rightleftarrows \text{res}_B(\psi_3,\psi_4) \rightleftarrows C\\
        \psi_6 \rightleftarrows \urcorner C (\rightleftarrows F_4)\\
        \psi_7 \rightleftarrows \text{res}_C(\psi_5,\psi_6)\rightleftarrows \varnothing
    \end{math}
    \underline{Замечание:} Из определения ясно, формулы $F_1,F_n, F$ - являются элементарными дизъюнктами\\
    \subsubsection*{Теорема 9(О полноте метода резолюций)}\label{th:7.9}
    Множества элементарных дизъюнкций противоречиво, тогда и только тогда, когда существует резолютивный вывод 
    $\varnothing$ из этого множества\par\noindent
    \underline{Метод резолюций}\\
    Задача: $F_1,\dots,F_n \models ? \; F_i - $ любая формула АВ\\
    \begin{enumerate}
        \item Найти КНФ каждой формулы $F_1,\dots,F_n$
        \item Раздробить каждую КНФ из п.1 на элементарные дизъюнкты
        \item Попробовать построить резолютивный вывод$\varnothing$ из множества элементарных дизъюнктов из п.2
    \end{enumerate}
    
    \begin{enumerate}
        \item если $\varnothing$ выводится, то $F_1,\dots,F_n \models$
        \item если $\varnothing$ не выводится, то $F_1,\dots,F_n \not \models$
    \end{enumerate}

    \begin{math}
        F_1 \rightleftarrows A \vee B \vee C \\
        F_2 \rightleftarrows A \rightarrow B \vee C \\
        F_3 \rightleftarrows \urcorner C\\
        A \rightleftarrows B
    \end{math}
    КНФ($F_1$) $\rightleftarrows A \vee B \vee C$\\
    КНФ($F_2$) $\rightleftarrows \urcorner A \vee B \vee C$\\
    КНФ($F_3$) $\rightleftarrows \urcorner C$\\
    КНФ($\urcorner F$) $\rightleftarrows \urcorner C$\\
\end{document}