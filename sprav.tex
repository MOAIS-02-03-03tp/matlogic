\documentclass[12pt]{article}
\usepackage[a4paper, total={7in,10in}]{geometry}
\usepackage{polyglossia}
\usepackage{ragged2e}
\usepackage{amsmath}
\usepackage{amssymb}
\usepackage{microtype}
\usepackage{graphicx}
\let\ORIincludegraphics\includegraphics
\renewcommand{\includegraphics}[2][]{\ORIincludegraphics[scale=0.65,#1]{#2}}
\usepackage{changepage}
\usepackage{hyperref}
\usepackage{cancel}
\graphicspath{{./images/}}
\setmainlanguage{russian}
\setotherlanguage{english}
\newfontfamily\russianfont[Script=Cyrillic]{Times New Roman}
\newfontfamily\englishfont{Times New Roman}
\setlength{\parindent}{0em}
\setlength{\parskip}{6pt}

\def\posl#1#2{\{#1_{#2}\}}
\DeclareMathOperator*{\sh-like}{\sinh-like}
\DeclareMathOperator*{\ch-like}{\cosh-like}
\DeclareMathOperator*{\th-like}{\tanh-like}
\DeclareMathOperator*{\cth-like}{\coth-like}
\DeclareMathOperator*{\tg-like}{\tan-like}
\DeclareMathOperator*{\ctg-like}{\cot-like}
\DeclareMathOperator*{\arctg-like}{\arctan-like}
\DeclareMathOperator*{\arcctg-like}{\arctan-like}

\begin{document}
    \pagebreak
    \tableofcontents
    \pagebreak
    \section{Теория множеств}
    \justifying
    \subsection{Основные определения}
    \begin{math}
        a \in A\\
        \varnothing\\
        A \subset B <=> A \subseteq B \; and \; \exists x (x \in B i x \not \in A)\\
        A \subseteq B <=> \forall x (x \in A => x \in B)\\
        A=B <=> A \subseteq B \; and \; B \subseteq A\\
        A \land B = {x|x \in A \; and \; x \in B}\\
        A \lor B = {x|x \in A \; or x \; \in B}\\
        A \setminus B={x|x \in A \; and \; x \not \in B}\\
        A \bigtriangleup B = (A \lor B) \setminus (A \land B)=
        (A \setminus B) \lor (B \setminus A)\\
        \bar{A_B}=B \setminus={x \in B | x \not \in A}\\
        A \; \text{x} \; \dots\; \text{x}\; A_n = {(a_1,\dots,a_n)|\forall i \in n (a_i \in A_i)}\\
        \text{Где $(a_1,\dots,a_n)$ - упорядоченный набор который определяется следующим образом}\\
    \end{math}
    \begin{enumerate}
        \item n=0 => $\varnothing$
        \item n=1 => {$a_1$}
        \item n=2 => ${{a_1},{a_1,a_2}}$
        \item n>2 => ($a_1,\dots,a_{n-1},a_n$)=($(a_1,\dots,a_{n-1}),a_n$)
    \end{enumerate}
    $(1,2,3)^{4)}=((1,2),3)^{3)}$=${{(1,2)},{3}}^{3}$={{{1},{1,2}},{(1,2),3}}=
    {{{{1},{1,2}}},{{{1},{1,2}},3}}

    \subsection{Бинарное отношение}
    \underline{Опр} Множество $\rho \subseteq A_1 x\dots x A_n$ называется n-местеыми отношениями на
    $(A_1,\dots,A_n)(n>0)$. Если $A_1=A_2=\dots=A_n(=A)$то $\rho$ называется n-местными отмношениями
    на A. Если n=2, то $\rho$ называется бинарным\\
    \underline{Пример}:\begin{enumerate}
        \item A={1,2,3}
        \item B={a,b}
        \item AxB={(1,a),(2,a),(3,a),(1,b),(2,b),(3,b)}
        \item BxA={(a,1),(a,2),(a,3),(b,1),(b,2),(b,3)}
    \end{enumerate}
    \underline{Опр} Пусть $\rho \subseteq AxB$\\
    Dom($\rho$)={$x\in A |\exists y \in b:(x,y)\in \rho$} - область определения\\
    Im ($\rho$)={$y \in B|\exists x \in A:(x,y)\in \rho$} - мн в значении\\
    $\rho^{-1}$ = {$(y,x)\in BxA|(x,y)\in \rho$} - обратное к $\rho$

    \subsection*{Свойства бинаных отношений на A}
    Пусть $\rho \subseteq A^2$. Тогда\\
    \begin{enumerate}
        \item $\rho$ рефлексивно <=> $\forall x \in A((x,x)\in \rho)$ definition
        \item $\rho$ симметрично <=> $\forall x,y \in A((x,y)\in \rho => (y,x)\in \rho)$
        \item $\rho$ транзитивно <=> $\forall x,y,z \in A ((x,y)\in \rho \; and \; (y,z)\in \rho => (x,z)\in \rho)$
        \item $\rho$ антирефлексивно <=> $\forall x \in A((x,x) \not \in \rho)$
        \item $\rho$ антисимметрично <=> $\forall x,y \in A (x \not = y \; and \; (x,y)\in \rho => (y,x)\not \in \rho)$
    \end{enumerate}
    \subsection{Отношение эквивалентности}
    \underline{Определение: }Бинарное отношение на множестве называется экивалентным, если оно 
    рефлексивно, симметрично, транзитивно.\\
    \underline{Пример:}\\
    \begin{enumerate}
        \item $\overset{\equiv}{5}$ на $\mathcal{Z}$\\
        x $\overset{\equiv}{5}$ y <=> x-y $\vdots$ 5\\
        \begin{enumerate}
            \item $x-y \vdots 5 => x \overset{\equiv}{5} x$ - рефлексивно
            \item $x \overset{\equiv}{5} y => x-y \vdots 5 => y-x \vdots 5 => y \overset{\equiv}{5}
            x $ - симметрично
            \item $\begin{matrix}
                x \overset{\equiv}{5} y => x-y \vdots 5\\
                y \overset{\equiv}{5} \mathcal{Z} => y - \mathcal{Z} \vdots 5
                \end{matrix} \Bigg\} x-\mathcal{Z} \vdots 5 => x \overset{\equiv}{5} \mathcal{Z}$ - транзитивно 
        \end{enumerate}
        \item A - ин-во все студентов в E702\\
        <x,y> $\in \rho \overset{def}{<=>}$ x и y сидят в одном ряду
        \item A - мн-во $\Delta $-ов пи-ти\\
        $<\Delta_1,\Delta_2> \in \rho \overset{def}{<=>} \Delta_1 \sim  \Delta_2$\\
        $\rho$ - отношение эквивалентности
    \end{enumerate}
    \underline{Определение: } Пусть $\rho$ - отн эквивалентности на A, a $\in$ A\\
    Мн-во $\frac{a}{\rho}=x \in A \Big|<a,x> \in \rho \Big| $ называется классом эквивалентности
    с представлением a.
    \subsubsection*{Теорема 1.3.1}\label{th:1.3.1} 
    $\sqsupset$ $\rho$ - отношение эквивалентности, на A, a,b $\in A$, тогда xa,by $\in \rho$
    <=> классом эл-ов a и b совпадают $\frac{a}{\rho}=\frac{b}{\rho}$\par\noindent
    \underline{Доказательство:}
    \begin{adjustwidth}{1.5em}{1.5em}
        $\Rightarrow \sqsupset x \in \frac{a}{\rho}$ по опр. <a,x> $\in \rho$. Т.к. <a,b>
        $\in \rho$ и $\rho$ асимметрично, то <b,a> $\in \rho$ $\overset{\text{транз $\rho$}}{\Rightarrow}$
        <b,x>$\in \rho \Rightarrow x \in \frac{b}{\rho} \Rightarrow \frac{a}{\rho}\leq \frac{b}{\rho}$,
        аналогично $\frac{b}{\rho}\leq\frac{a}{\rho} \Rightarrow \frac{a}{\rho}=\frac{b}{\rho}$\\
        $\Leftarrow$ Т.к. $\rho$ рефлексивно, то <a,a> $\in \rho$, т.е. $a \in \frac{a}{\rho}$. Из
        $\frac{a}{\rho}=\frac{b}{\rho} \Rightarrow 2 \in \frac{b}{\rho}$, поэтому <b,a>$\in \rho
        \Rightarrow <a,b> \in \rho$
    \end{adjustwidth}
    \subsection{Отношение порядка}
    \underline{Определение: } Бинарное отношение $\rho$ на мн-ве A называется отношением
    порядка, если оно антиасимметрично и транзитивно.
    \begin{enumerate}
        \item Если $\rho$ рефлексивно то оно называется не строгим
        \item Если $\rho$ антирефлексивно то строгим порядком
        \item Если $\rho$ связно, то линейным порядком
    \end{enumerate}
    \underline{Определение: } Если на мн-ве A введено отношение порядка $\rho$, то A
    называют упорядоченным с помощью $\rho$ мн-вом.\\
    \underline{Пример:} $\leq$ на $\mathcal{Z}$
    \begin{enumerate}
        \item <x,y> $\in \leq <=>(\exists t \in N \; x+\mathcal{Z}=y)$ или x=y
        \begin{enumerate}
            \item $\left.
              \begin{aligned}
                x \leq y \\
                y \leq x
              \end{aligned}
              \right\} \Rightarrow x = y$ - антисимметричность
            \item $\left.
              \begin{aligned}
                x \leq y \\
                y \leq t
              \end{aligned}
              \right\} \Rightarrow x \leq t$ - транзитивность\\
              Оба пункта образуют порядок
            \item $\psi \leq x$ => не строгий порядок
            \item $\psi \not =$ y => $x \leq y$ или $y \leq x$ => линейный
        \end{enumerate}
        \item Рассмотрим на $\rho(\mathcal{A}),$ где $\mathcal{A}=<a,b>$\\
        <x,y> $\in \leq <=> x \leq y,$ где $x,y \in \mathcal{A}$
        \begin{enumerate}
            \item $\begin{aligned}
                x \leq y\\
                y \leq \psi
            \end{aligned}\Bigg\} \psi \leq y$
            \item $\begin{aligned}
                x \leq y\\
                y \leq \mathcal{Z}
            \end{aligned}\Bigg\} x \leq \mathcal{Z}$
            \item $x \leq \psi \Rightarrow$ не строгий порядок
            \item $\{a\} \not = \{b\},$ на $\{a\} \leq \{b\}$ и $\{b\} \leq \{a\} \Rightarrow $ не линейный
        \end{enumerate}
    \end{enumerate}
    \underline{Определение: }$\sqsupset \rho-$ порядок на A, $a \in A,b \in A$
    \begin{enumerate}
        \item a - минимальный $\overset{def}{<=>} \forall x \in A (<x,a> \in \rho \Rightarrow x =a)$
        \item a - максимальный $\overset{def}{<=>} \forall x \in A (<a,x> \in \rho \Rightarrow x =a)$
        \item a - наименьший $\overset{def}{<=>} \forall x \in A (a \not = x \Rightarrow <a,x> \in \rho)$
        \item a - наибольший $\overset{def}{<=>} \forall x \in A (a \not = x \Rightarrow <x,a> \in \rho)$
    \end{enumerate}
\end{document}